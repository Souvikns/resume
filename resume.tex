\documentclass[10pt, letterpaper]{article}

% Packages:
\usepackage[
    ignoreheadfoot, % set margins without considering header and footer
    top=2 cm, % seperation between body and page edge from the top
    bottom=2 cm, % seperation between body and page edge from the bottom
    left=2 cm, % seperation between body and page edge from the left
    right=2 cm, % seperation between body and page edge from the right
    footskip=1.0 cm, % seperation between body and footer
    % showframe % for debugging 
]{geometry} % for adjusting page geometry
\usepackage{titlesec} % for customizing section titles
\usepackage{tabularx} % for making tables with fixed width columns
\usepackage{array} % tabularx requires this
\usepackage[dvipsnames]{xcolor} % for coloring text
\definecolor{primaryColor}{RGB}{0, 0, 0} % define primary color
\usepackage{enumitem} % for customizing lists
\usepackage{fontawesome5} % for using icons
\usepackage{amsmath} % for math
\usepackage[
    pdftitle={Souvik De's CV},
    pdfauthor={Souvik De},
    pdfcreator={LaTeX with RenderCV},
    colorlinks=true,
    urlcolor=primaryColor
]{hyperref} % for links, metadata and bookmarks
\usepackage[pscoord]{eso-pic} % for floating text on the page
\usepackage{calc} % for calculating lengths
\usepackage{bookmark} % for bookmarks
\usepackage{lastpage} % for getting the total number of pages
\usepackage{changepage} % for one column entries (adjustwidth environment)
\usepackage{paracol} % for two and three column entries
\usepackage{ifthen} % for conditional statements
\usepackage{needspace} % for avoiding page brake right after the section title
\usepackage{iftex} % check if engine is pdflatex, xetex or luatex

% Ensure that generate pdf is machine readable/ATS parsable:
\ifPDFTeX
    \input{glyphtounicode}
    \pdfgentounicode=1
    \usepackage[T1]{fontenc}
    \usepackage[utf8]{inputenc}
    \usepackage{lmodern}
\fi

\usepackage{charter}

% Some settings:
\raggedright
\AtBeginEnvironment{adjustwidth}{\partopsep0pt} % remove space before adjustwidth environment
\pagestyle{empty} % no header or footer
\setcounter{secnumdepth}{0} % no section numbering
\setlength{\parindent}{0pt} % no indentation
\setlength{\topskip}{0pt} % no top skip
\setlength{\columnsep}{0.15cm} % set column seperation
\pagenumbering{gobble} % no page numbering

\titleformat{\section}{\needspace{4\baselineskip}\bfseries\large}{}{0pt}{}[\vspace{1pt}\titlerule]

\titlespacing{\section}{
    % left space:
    -1pt
}{
    % top space:
    0.3 cm
}{
    % bottom space:
    0.2 cm
} % section title spacing

\renewcommand\labelitemi{$\vcenter{\hbox{\small$\bullet$}}$} % custom bullet points
\newenvironment{highlights}{
    \begin{itemize}[
        topsep=0.10 cm,
        parsep=0.10 cm,
        partopsep=0pt,
        itemsep=0pt,
        leftmargin=0 cm + 10pt
    ]
}{
    \end{itemize}
} % new environment for highlights


\newenvironment{highlightsforbulletentries}{
    \begin{itemize}[
        topsep=0.10 cm,
        parsep=0.10 cm,
        partopsep=0pt,
        itemsep=0pt,
        leftmargin=10pt
    ]
}{
    \end{itemize}
} % new environment for highlights for bullet entries

\newenvironment{onecolentry}{
    \begin{adjustwidth}{
        0 cm + 0.00001 cm
    }{
        0 cm + 0.00001 cm
    }
}{
    \end{adjustwidth}
} % new environment for one column entries

\newenvironment{twocolentry}[2][]{
    \onecolentry
    \def\secondColumn{#2}
    \setcolumnwidth{\fill, 4.5 cm}
    \begin{paracol}{2}
}{
    \switchcolumn \raggedleft \secondColumn
    \end{paracol}
    \endonecolentry
} % new environment for two column entries

\newenvironment{threecolentry}[3][]{
    \onecolentry
    \def\thirdColumn{#3}
    \setcolumnwidth{, \fill, 4.5 cm}
    \begin{paracol}{3}
    {\raggedright #2} \switchcolumn
}{
    \switchcolumn \raggedleft \thirdColumn
    \end{paracol}
    \endonecolentry
} % new environment for three column entries

\newenvironment{header}{
    \setlength{\topsep}{0pt}\par\kern\topsep\centering\linespread{1.5}
}{
    \par\kern\topsep
} % new environment for the header

\newcommand{\placelastupdatedtext}{% \placetextbox{<horizontal pos>}{<vertical pos>}{<stuff>}
  \AddToShipoutPictureFG*{% Add <stuff> to current page foreground
    \put(
        \LenToUnit{\paperwidth-2 cm-0 cm+0.05cm},
        \LenToUnit{\paperheight-1.0 cm}
    ){\vtop{{\null}\makebox[0pt][c]{
        \small\color{gray}\textit{Last updated in September 2024}\hspace{\widthof{Last updated in September 2024}}
    }}}%
  }%
}%

% save the original href command in a new command:
\let\hrefWithoutArrow\href

% new command for external links:


\begin{document}
    \newcommand{\AND}{\unskip
        \cleaders\copy\ANDbox\hskip\wd\ANDbox
        \ignorespaces
    }
    \newsavebox\ANDbox
    \sbox\ANDbox{$|$}

    \begin{header}
        \fontsize{25 pt}{25 pt}\selectfont Souvik De

        \vspace{5 pt}

        \normalsize
        \mbox{India}%
        \kern 5.0 pt%
        \AND%
        \kern 5.0 pt%
        \mbox{\hrefWithoutArrow{mailto:souvikde.ns@gmail.com}{souvikde.ns@gmail.com}}%
        \kern 5.0 pt%
        \AND%
        \kern 5.0 pt%
        \mbox{\hrefWithoutArrow{tel:+91-9073302976}{+91 9073302976}}%
        \kern 5.0 pt%
        \AND%
        \kern 5.0 pt%
        \mbox{\hrefWithoutArrow{https://souvikns.com/}{souvikns.com}}%
        \kern 5.0 pt%
        \AND%
        \kern 5.0 pt%
        \mbox{\hrefWithoutArrow{https://www.linkedin.com/in/souvik-de-a2b941169/}{linkedin}}%
        \kern 5.0 pt%
        \AND%
        \kern 5.0 pt%
        \mbox{\hrefWithoutArrow{https://github.com/Souvikns/}{github/Souvikns}}%
    \end{header}

    \vspace{5 pt - 0.3 cm}


    \section{Summary}



        
        \begin{onecolentry}
            Software Developer with 3+ years of professional experience specializing in API development. Previously worked at \textbf{Postman} as a Software Developer, where I maintained \textbf{AsyncAPI} and integrated AsyncAPI specifications into Postman's pipelines. Currently working at \textbf{Codemate.AI}, where I've built and shipped a RAG-based AI agent for code generation. Active open-source contributor serving as Core Maintainer and Technical Steering Committee member at AsyncAPI, where I manage two projects with approximately 16,000 monthly NPM downloads. Contributed to developer community growth by mentoring in Google Summer of Code 2023 under Postman.
        \end{onecolentry}

    
    \section{Experience}

        
        \begin{twocolentry}{
            December 2024 – May 2025
        }
            \textbf{Lead Software Engineer}, CodeMate.AI -- Noida\end{twocolentry}

        \vspace{0.10 cm}
        \begin{onecolentry}
            \begin{highlights}
                \item Engineered an enhanced search algorithm for Swagger files, optimizing RAG system performance and search precision.
                \item Implemented CI/CD pipeline utilizing GitHub Actions to automate application release cycles, enhancing deployment efficiency and reliability.
                
                \item Optimized a local server utilized by our VS Code extension, achieving over 30\% improvement in startup time and enhancing overall user experience.

                \item Implemented automated deployment from GitHub Actions to a virtual machine (VM) and configured branch protection rules, improving code quality, deployment stability, and increasing team productivity by eliminating manual deployment efforts.
                
            \end{highlights}
        \end{onecolentry}


        \vspace{0.2 cm}

        \begin{twocolentry}{
            Feb 2022 – June 2024
        }
            \textbf{Software Engineer}, Postman -- Bangalore\end{twocolentry}

        \vspace{0.10 cm}
        \begin{onecolentry}
            \begin{highlights}
                \item Developed and maintained a CLI application that integrated all official tools supported by the AsyncAPI Initiative, providing a unified solution for developers working with the AsyncAPI specification.
                
                \item Initiated and developed AsyncAPI Bundler, an npm package that intelligently resolves \texttt{\$ref} references and merges multiple AsyncAPI specification files. The tool has gained widespread adoption with over 16,000 downloads.

                \item Serve as a maintainer for over three open-source projects, leading development efforts, reviewing pull requests, and driving project direction within the community.

                \item Served as a mentor for Google Summer of Code (GSoC) under Postman, successfully guiding participants to complete their projects and achieve program goals.

                \item Initiated and led a working group with a colleague to define a common governance framework for building parsers and to identify tooling that minimizes development time and effort across multiple programming languages.
                
                
            \end{highlights}
        \end{onecolentry}

        \vspace{0.2 cm}

        \begin{twocolentry}{
            Nov 2021 – Feb 2022
        }
            \textbf{Backend Developer Intern}, Mage -- Remote\end{twocolentry}

        \vspace{0.10 cm}
        \begin{onecolentry}
            \begin{highlights}
                \item Developed microservices to automate cloud infrastructure provisioning using Go’s templating engine, improving scalability and deployment efficiency.

                \item Built a React-based web application to deliver a low-code solution for generating and provisioning cloud infrastructure on demand.

                \item Gained hands-on experience with Docker and Kubernetes to deploy, manage, and scale microservices effectively.
                
            \end{highlights}
        \end{onecolentry}
    
    \section{Projects}



        
        \begin{twocolentry}{
            \href{https://github.com/Souvikns/Notion-Board}{Notion-Board}
        }
            \textbf{GitHub Action to sync GitHub Issues to Notion Database}\end{twocolentry}

        \vspace{0.10 cm}
        \begin{onecolentry}
            \begin{highlights}
                \item Developed a GitHub Action to auto sync and update github issues and it's state with Notion Pages.
                \item Earned recognition from the open-source community through project stars and valuable user feedback.
                \item Tools Used: Typescript, GH Action.
                
            \end{highlights}
        \end{onecolentry}


        \begin{twocolentry}{
            \href{https://github.com/Souvikns/asyncapi-rag}{AsyncAPI-RAG}
        }
            \textbf{FastAPI application that scrapes AsyncAPI documentation to create a RAG system and provides a chat API to answer user queries.}\end{twocolentry}

        \vspace{0.10 cm}
        \begin{onecolentry}
            \begin{highlights}
                \item Built data scrapper to scrape AsyncAPI spec and tools documentation from github.
                \item Used LangChain to chunk Markdown data, generated embeddings with the Nomic embed model from Ollama, stored embeddings in a Qdrant database, and exposed a REST API to handle user queries and return contextual answers.
                \item Tools Used: Python, qdrant, Langchain, ollama, docker 
                
            \end{highlights}
        \end{onecolentry}

    
    \section{Technologies}



        
        \begin{onecolentry}
            \textbf{Languages:} Javascript, Typescript, Go, Python, SQL
        \end{onecolentry}

        \vspace{0.2 cm}

        \begin{onecolentry}
            \textbf{Technologies:} NodeJs, FastAPI, Gin, Postgress, Firebase, NextJS, React, Langchain, Oclif 
        \end{onecolentry}

    
    \section{Education}

        
        \begin{twocolentry}{
            June 2018 – June 2022
        }
            \textbf{Chandigarh University}, BE in Computer Science\end{twocolentry}

        \vspace{0.10 cm}
        \begin{onecolentry}
            \begin{highlights}
                \item \textbf{Coursework:} Computer Architecture, Comparison of Learning Algorithms, Computational Theory
            \end{highlights}
        \end{onecolentry}



    

\end{document}